\documentclass[twocolumn]{article}
\usepackage[toc,page]{appendix}
\usepackage{amsmath}
\title{Thermoshit of Rocket Fuckups during disasterous flight}
\author{Eric Souder}
\date{July 2022}
\twocolumn

\begin{document}
    \maketitle
    \begin{abstract}
    \end{abstract}
    \section{Introduction}
        During the launch of a sounding rocket, the exterior of the vehicle experiences
        heat loading, etc etc idk

    \section{Heating of the Nosecone Skin}
        As the vehicle travels at speed through the atmosphere during launch, it
        will experience heating both from solar radiation and 
        aerodynamic heating as the vehicle compresses the air in front of it.
        \subsection{Nosecone Aerodynamic Heating}
            As the nosecone travels at suspersonic speeds, it forms a shock wave
            as it moves faster than the air can escape. This can also be
            considered as air being blown towards a stationary nosecone, at the
            speed the rocket would be travling. At the very tip of
            the nosecone, the air has zero relative velocity to the vehicle. 
            This is the stagnation point [REF!].

            At the stagnation point, all of the kinetic energy of the air is 
            converted into thermal energy. Becuase this process happens so
            quickly, it can be modeled as an adiabatic compression of gas.

            \[PV^\gamma = \textrm{constant}\]

            The ideal gas approximation ($V\propto TP^{-1}$) can be applied, so 

            \[P^{1-\gamma}T^\gamma=\textrm{constant}\]

            In our compression, this means 

            \[\frac{T_s}{T_0}=\left(\frac{P_s}{P_0}\right)^\frac{\gamma-1}{\gamma}\]

            With $T_s, P_s$ as the stagnation temperature and pressure and 
            $T_0, P_0$ as the static temperature and pressure. The relation 
            between the static and stagnation pressure of a gas is sourced from 
            a National Adivosry Committee for Aeronautics report on compressable 
            flow:
            %https://www.grc.nasa.gov/WWW/BGH/Images/naca1135.pdf, eqn 44%

            \[\frac{P_s}{P_0} =\left(1+\frac{\gamma-1}{2}M^2\right)^{\frac{\gamma}{\gamma-1}}\]

            Where $M$ is the mach number. This provides an equation for
            stagnation temperature as a function of speed:

            \[\frac{T_s}{T_0}=1+\frac{\gamma-1}{2}M^2\]

            Of course, the entire nose cone will not encounter air at the
            stagnation temperature. Most air will be slowed, but stopped by skin 
            friction with the vehicle. Toft [citatation!]
            %http://dark.dk/documents/technical_notes/simplified%20aerodynamic%20heating%20of%20rockets.pdf%
            %might be more accurate to use NACA or something citation here, see toft's citation%
            models the temperature of the boundary layer air ($T_B$) based on
            $K$, a temperature recovery factor:

            \[K=\frac{T_B-T_0}{T_s-T_0}\]

            Eber [citation] provides a value of $K=0.89$ for cones with vertex
            angles between 20 and 50 degrees.

            We then have an equation for the temperature of the boundary layer:

            \[T_B=KT_s+T_0\left(1-K\right)\]

            Eber also provides an experimentally modeled value for $h$, the heat
            transfer function:
            
            \[h=\left(0.0071+0.0154 \sqrt[2]{\beta}\right)\frac{k}{\mu^{0.8}l^{0.2}}\left(\rho_0 u\right)\]

            based on $k$, the thermal conductivity of air; $\beta$, the vertex 
            angle of the cone; $\rho$, the density of the air; $\mu$, the dynamic
            viscosity of air; $u$ the velocity of the rocket; and $l$, the 
            length of the cone measured along the surface. 

            With all this, we can determine the heat flux into the nosecone skin
            from aerodynamic effects:

            \begin{equation}
                \dot{Q}_{aero}= h(T_B-T_N)
            \end{equation}

            Where $T_N$ is the actual temperaure of the nosecone skin.

        \subsection{Radiative Effects}
            Some amount of heat flux leaves the nosecone as blackbody radiation,
            with heat flux $\dot{Q}_rad=\epsilon\sigma(T_0^4-T_N^4)$, where
            $\epsilon$ is the emmisivity of stainless steel nosecone and 
            $\sigma$ is the Stefan-Boltzman constant.

            %talk about absorption of solar radiation?%

        \subsection{Flight Profile}
            In order to model the thermal behavior of the nosecone, we must
            provide a number of inputs to our aerodynamic heating simulation 
            functions based on altitude and velocity. These inputs are provided 
            by UBC Rocket's proprietary Feynman engine design program, which 
            outputs the altitude and Mach number of a simulated rocket flight.

            Model curves of $\rho_0$ and  $T_0$ in terms of altitude and of 
            $\mu$ and $K$ in terms of $T_0$ are provided in appendix \ref{appendix:a}
        \subsection{Heat Flux Balance}
            The nosecone situation through flight can be modeled by an energy 
            balance:

            \[\textrm{Heat Flux In} = \textrm{Heat Flux out}\]

            This can be modeled by the heat equation:

            \[G dT_N = dt(\dot{Q}_{aero}-\dot{Q}_{rad})\]

            With a factor $G$, the 'skin heating capacity' [cite] % note cite toft's [1] source or toft%
            determined by the specific heat of the nosecone skin $c$, it's 
            thickness $\tau$, and it's density $\rho$.

            \[G=c\tau\rho\]

        \subsection{Simulation}

            Considering a small but finite change in time $\Delta T$, we can
            numerically solve for the change $\Delta T_N$ in the nosecone
            temperature.

            This is accomplished using a python simulation.



    \onecolumn
    \begin{appendices}
        \section{Atmospheric Models}
        \label{appendix:a}
        \subsection{Air Pressure as a function of Altitude}
            Air pressure $P_0$ as a funciton of altitude $s$ [reference]
            \[P_0(s)=\begin{cases}
                \exp{\left(4.43165\cdot10^{-14}s^3-2.28553\cdot10^{-9}s^2-1.14097\cdot10^{-4}s+6.95109\right)} & 0\textrm{km}<s\leq 25\textrm{km}\\
                \exp{\left(-2.28179\cdot10^{-14}s^3+3.34063\cdot10^-9s^2-2.84655\cdot10^{-4}s+8.73033\right)} & 25\textrm{km} < s \leq 75\textrm{km}\\
                \exp{\left(4.44813\cdot10^{-14}s^3-1.13434\cdot10^-9s^2+7.62651\cdot10^{-4}s-15.5981\right)} & 75\textrm{km} < s \leq 120\textrm{km}
            \end{cases}\]
        \section{Simulation Constants}
    \end{appendices}

















\end{document}